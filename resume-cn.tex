% !TEX program = xelatex
% This is my resume
% Chinese translation
% by ice1000
% modified by yxsong

\documentclass{resume}

\usepackage{lastpage}
\usepackage{fancyhdr}
\usepackage{linespacing_fix} % disable extra space before next section
\usepackage[fallback]{xeCJK}

%% \setmainfont[]{SimSun}
%% \setCJKfallbackfamilyfont{rm}{HAN NOM B}
% \setCJKmainfont[BoldFont=SimHei,ItalicFont=KaiTi_GB2312]{SimSun}
%% \renewcommand{\thepage}{\Chinese{page}}

\begin{document}
\pagestyle{fancy}
\fancyhf{}
\renewcommand\headrulewidth{0pt}
\cfoot{\thepage\ of \pageref{LastPage}}

\name{宋英旭}

\basicInfo{
  \email{yxsong@cug.edu.cn} \textperiodcentered\ 
  \phone{(+86)13163337265} \textperiodcentered\ 
  \github[Yinghsusong]{https://github.com/Yinghsusong}
  % \CSDN[SONGYINGXU]{https://blog.csdn.net/SONGYINGXU}
}

\section{\faGraduationCap\ 教育经历}
\datedsubsection{\textbf{中国地质大学(武汉)}}{2012.9 -- 现在}
  专业:地球探测与信息技术,硕博连读,预计毕业日期:2019.6
\datedsubsection{\textbf{中国地质大学(武汉)}}{2008.9 -- 2012.6}
  专业:地球信息科学与技术,本科

\section{\faUsers\ 工作经历}
\datedsubsection{\textbf{地震局地壳所}, 北京, 中国}{2015.8 -- 2016.11}
\role{学习交流}{干涉合成孔径雷达(InSAR)理论学习,地震数据处理}
\begin{itemize}
  \item 学习InSAR理论知识和微波遥感知识
  \item 处理多个地震的同震形变场,包括Bam地震、台湾美浓地震、尼泊尔地震以及当雄地震等
  \item 学到了微波干涉相关的许多理论知识和地震同震形变场的提取和处理方法
\end{itemize}

\section{\faGithubAlt\ 项目经历}

\datedsubsection{\textbf{重大工程地质灾害快速监测与评估(国家高技术研究发展
计划“863”项目)}}{}
\begin{itemize}
  \item 负责开发“滑坡灾害遥感监测与评估系统”及相关文档的编写
  \item 该软件系统主要以三峡库区滑坡地质灾害为主要示范对象,将滑坡敏感性制图过程工程化、快速化
  \item 为库区灾害防治与预警指挥提供参考依据
\end{itemize}

\datedsubsection{\textbf{地质环境遥感监测应用子系统(中国地质环境监测院)}}{}
\begin{itemize}
  \item 负责开发“地质环境遥感监测应用子系统”及相关项目文档的编写
  \item 该软件系统以滑坡、崩塌、泥石流等地质灾害为主要研究对象,使用流程化工具进行致灾因子的提取和处理,为地质环境监测提供依据
\end{itemize}

\datedsubsection{\textbf{三峡库区地理数据库建设及遥感解译(三峡库区指挥中心)}}{}
\begin{itemize}
  \item 主要负责研究高分影像正射影像制作方法与流程
  \item 研究道路、地表覆盖物的自动解译方法
  \item 对人工解译废渣堆、库区典型地质灾害(滑坡)进行人工解译
\end{itemize}

\datedsubsection{\textbf{武忠管道地质灾害风险评估建模及系统研发(中国石油天然气股份有限公司管道分公司管道科技研究中心)}}{}
\begin{itemize}
  \item 负责系统开发及相关文档编写(C\#/ArcEngine)
  \item 以工程地质和水文地质为基础进行滑坡、崩塌的稳定性和破坏概率的计算
  \item 利用GIS 技术、RS 技术等 3S 技术进行区域地质灾害风险评估,进而实现点面结合的管道沿线地质灾害监测与评估
\end{itemize}

\datedsubsection{\textbf{基于国产资源卫星的地质灾害遥感监测关键技术研究}}{}
\begin{itemize}
  \item 开展重庆“8.31”暴雨滑坡解译工作
  \item 基于Google Earth Engine云平台开发滑坡地质灾害滑坡易发性风险动态评价系统
  \item 编写项目相关文档
\end{itemize}


\section{\faUsers\ 发表文章}
\begin{itemize}

\item  Song, Y.; Niu, R.; Xu, S.; Ye, R.; Peng, L.; Guo, T.; Li, S.; Chen, T. Landslide Susceptibility Mapping Based on Weighted Gradient Boosting Decision Tree in Wanzhou Section of the Three Gorges Reservoir Area (China). ISPRS Int. J. Geo-Inf. 2019, 8, 4.
\item 宋英旭, 牛瑞卿, 张景发,等. 遥感影像变化检测方法对比[J]. 地壳构造与地壳应力文集, 2016(2).
\item Song Y, Niu R, Zhang J, et al. The Deformation Measurement of Laohushan Fault Based on the FRAM-SBAS Method[C]// Dragon 3 Final Results and Dragon 4 Kick-Off. Dragon 3 Final Results and Dragon 4 Kick-Off, 2016.
\end{itemize}

\section{\faCogs\ 个人技能}
\begin{itemize}[parsep=0.25ex]
  \item \textbf{编程语言}:
    比较熟悉 Java/Python/C\#/Matlab,
    一般熟悉 C++/C/SQL/Java Script/Tex
  
  \item \textbf{专业技能}:
  熟练掌握开源 GIS(QGIS、Geoserver)、商业 GIS 和遥感软件(ArcGIS、MapGIS、eCognition、ENVI)、以及 Weka 等数据挖掘软件的使用和操作;熟悉遥感影像及 GIS 数据处理、遥感影像解译、制图流程

  \item \textbf{计算机}:
  获计算机技术与软件专业技术资格(水平)初级职称;取得“基于 GEE 云平台的滑坡地质灾害风险评价平台”和“基于 ArcGIS Engine 的管道地质灾害监测预警平台”软件著作权两项

\end{itemize}

\section{\faHeartO\ 获奖经历}
\datedline{湖北省大学生象棋锦标赛一等奖}{2015}
\datedline{湖北省高校“学府杯”象棋比赛团体第一名}{2015}
\datedline{湖北省大学生象棋竞标赛三等奖}{2017}
\datedline{地空学院手抄党章书法比赛一等奖}{2017}
\datedline{全国研究生数学建模比赛优秀奖}{2017}
\datedline{第六届“东方杯”全国大学生勘探地球物理大赛优秀奖}{2018}

\section{\faInfo\ 其他}
\begin{itemize}[parsep=0.25ex]
  \item 博客: \url{https://blog.csdn.net/SONGYINGXU} 
  \item 毕设: 基于空天地一体化监测的滑坡易发性动态评价研究
  \item 职务: 2017.9-2018.6,担任2014级地空学院博士党支部书记
  \item 社会地理计算: 有幸向新泽西理工大学的叶信岳老师学习关于社会地理计算和网络分析的知识,使用过叶信岳老师团队开源的TrajVis软件和SocialNetworkSimulator软件,并提交过软件安装流程和反馈意见
  \item 社会地理: 之前做的较多的是自然地理,通过跟叶老师的学习,对人文地理和社会地理十分感兴趣,拟开展基于夜间灯光遥感数据和网络分析相关的研究
  \item 大数据云计算:对于大数据云平台的相关知识比较感兴趣,对于Google开源的GEE大数据云平台有一定研究,有使用Spark分布式平台进行基于LSTM模型进行自然语言处理的经验,希望我国能在大数据云计算推广方面迎头赶上,也希望自己能在我国云平台建设的过程中贡献自己的力量
\end{itemize}

%% Reference
%\newpage
%\bibliographystyle{IEEETran}
%\bibliography{mycite}
\end{document}
