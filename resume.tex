% !TEX program = xelatex
% This is my resume
% by ice1000

\documentclass{resume}

\usepackage{lastpage}
\usepackage{fancyhdr}
\usepackage{linespacing_fix} % disable extra space before next section

\begin{document}
\pagestyle{fancy}
\fancyhf{}
\renewcommand\headrulewidth{0pt}
\cfoot{\thepage\ of \pageref{LastPage}}

\name{Song Ying Xu}

\basicInfo{
  \email{yxsong@cug.edu.cn} \textperiodcentered\ 
  \phone{(+86)13163337265} \textperiodcentered\ 
  \github[Yinghsusong]{https://github.com/Yinghsusong}
  % \CSDN[SONGYINGXU]{https://blog.csdn.net/SONGYINGXU}
}

\section{\faGraduationCap\ Education}
\datedsubsection{\textbf{China University of Geosciences (Wuhan))}, Hubei, China}{09/2012 -- Current}
  Major: Earth Exploration and Information Technology, Master \& Doctor, Anticipated Graduation: 06/2019

\datedsubsection{\textbf{China University of Geosciences (Wuhan))}, Hubei, China}{09/2008 -- 06/2012}
  Major: Earth Information Science and Technology, Bachelor

\section{\faUsers\ Work Experience}
\datedsubsection{\textbf{The Institude of Crustal Dynamics}, Beijing, China}{08/2015 -- 11/2016}
\role{Exchange Student}{Interferometric Synthetic Aperture Radar (InSAR) theory learning, seismic data processing}
\begin{itemize}
  \item Learned InSAR theoretical knowledge and microwave remote sensing knowledge
  \item processed coseismic deformation fields of multiple earthquakes
  \item Learned a lot about Linux programming and system
\end{itemize}

\section{\faUsers\ Articles}
\begin{itemize}

\item  Song Y., Niu R., Xu S., et al. Landslide Susceptibility Mapping Based on Weighted Gradient Boosting Decision Tree in Wanzhou Section of the Three Gorges Reservoir Area (China)[J]. ISPRS Int. Geo-Inf. 2019, 8, 4.
\item Song Y., Niu R., Zhang J., et al. Comparison of Change Detection Methods in Remote Sensing [J]. Bulletin of the Institude of Crustal Dynamics, 2016(2). (In Chinese)
\item Song Y., Niu R., Zhang J., et al. The Deformation Measurement of Laohushan Fault Based on the FRAM-SBAS Method[C]// Dragon 3 Final Results and Dragon 4 Kick-Off. Dragon 3 Final Results and Dragon 4 Kick-Off, 2016.
\end{itemize}

\section{\faGithubAlt\ Projects Experience}

\datedsubsection{\textbf{Rapid monitoring and evaluation of major engineering geological disasters (National High-Tech Research and Development Program “863” project)}}{}

Python progamming language running on QGIS.
\begin{itemize}
  \item Responsible for the development of “Landslide Hazard Risk Monitoring and Evaluation System”
  \item The software system mainly used landslide geological disasters in the Three Gorges Reservoir Area
  \item The main demonstration object was to engineer and speed up the landslide sensitivity mapping process and provided reference for disaster prevention and early warning in the reservoir area
\end{itemize}

\datedsubsection{\textbf{Geological environment remote sensing monitoring application subsystem (China Geological Environment Monitoring Institute)}}{}
C++/Qt progamming language.

\begin{itemize}
  \item Responsible for the development of the software system
  \item The software system took geological disasters such as landslides, collapses and mudslides as the main research objects, and used "pipeline" tools to extract and process the hazard factors, providing a basis for geological environment monitoring
\end{itemize}

\datedsubsection{\textbf{Geodatabase Construction and Remote Sensing Interpretation in the Three Gorges Reservoir Area (Three Gorges Reservoir Area Command Center)}}{}

\begin{itemize}
  \item Mainly responsible for researching high-resolution image orthophoto production methods and processes
  \item Research on automatic interpretation methods for roads and surface coverings
  \item Manual interpretation of artificially interpreted waste slag piles and typical geological disasters (landslides) in the reservoir area
\end{itemize}

\datedsubsection{\textbf{Wuzhong Pipeline Geological Disaster Risk Assessment Modeling and System Development}}{}

C\# progamming language running on ArcEngine.

\begin{itemize}
  \item Responsible for system development
  \item Calculation of stability and failure probability of landslides and collapses based on engineering geology and hydrogeology
  \item Using GIS, RS and other 3S technologies to carry out regional geological disaster risk assessment, and then realize the monitoring and evaluation of geological disasters along the pipeline
\end{itemize}

\datedsubsection{\textbf{Research on Key Technologies of Remote Sensing Monitoring of Geological Hazards Based on Domestic Resource Satellite}}{}

Java Script progamming language running on Google Earth Engine.

\begin{itemize}
  \item Carry out the interpretation of the “8.31” rainstorm landslide in Chongqing
  \item Development of landslide geological disaster landslide susceptibility risk dynamic evaluation system based on Google Earth Engine cloud platform
  \item Write project related documents
\end{itemize}
% Reference Test
%\datedsubsection{\textbf{Paper Title\cite{zaharia2012resilient}}}{May. 2015}
%An xxx optimized for xxx\cite{verma2015large}
%\begin{itemize}
%  \item main contribution
%\end{itemize}

\section{\faHeartO\ Achievements}
\datedline{First Prize of Hubei University Student Chinese Chess Championship}{2015}
\datedline{Third Prize of Hubei University Student Chinese Chess Championship}{2017}
\datedline{National Graduate Mathematical Modeling Competition Excellence Award}{2017}
\datedline{The 6th "Oriental Cup" National University Students Exploration Geophysical Competition Excellence Award}{2018}

\section{\faCogs\ Skills}
\begin{itemize}[parsep=0.25ex]
  \item \textbf{Programming Languages}:
    \textbf{multilingual developer} (not limited to any specific language),
    and especially experienced in Java/Python/C\#/Matlab,
    comfortable with C/C++/Java Script/Tex

  \item \textbf{Software}:
    ArcGIS, ENVI, SARscape, Matlab, SPSS, QGIS
  
  % platforms I am familiar with
  \item \textbf{Developing Tools}:
    can adapt to any editors/OSs, usually use Visual Studio Code or Visual Studio 2015 under Windows 10
\end{itemize}

% \section{\faHeartO\ Honors and Awards}
% \datedline{\textit{\nth{1} Prize}, Award on xxx }{Jun. 2013}
% \datedline{Other awards}{2015}

% \section{\faInfo\ Miscellaneous}
% \begin{itemize}[parsep=0.25ex]
%   \item Love making friends
% \end{itemize}

%% Reference
%\newpage
%\bibliographystyle{IEEETran}
%\bibliography{mycite}
\end{document}

